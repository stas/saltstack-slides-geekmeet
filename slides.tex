\documentclass[compress]{beamer}

\beamertemplatetransparentcovered
\setbeamertemplate{navigation symbols}{}
\setbeamertemplate{footline}
{\leavevmode%
  \hbox{%
    \begin{beamercolorbox}[wd=.3333\paperwidth,left]{}%
      \insertshortinstitute
    \end{beamercolorbox}%
    \begin{beamercolorbox}[wd=.3333\paperwidth,center]{}%
      \insertframenumber{} / \inserttotalframenumber
    \end{beamercolorbox}
    \begin{beamercolorbox}[wd=.3333\paperwidth,right]{}%
      \insertshortdate{}
      \;
    \end{beamercolorbox}
  }%
}

\usepackage{listings}
\usepackage{color}
\usepackage{minted}
\usepackage{hyperref}
\usepackage{textcomp}
\usepackage{marvosym}
\usepackage{xcolor}
\usepackage{graphicx}

\usetheme[light,accent=violet]{solarized}

\usepackage{fontspec}
\setmainfont[Mapping=tex-text]{Source Sans Pro}
\setmonofont[Mapping=tex-text]{Inconsolata}

\graphicspath{{./images/}}

\title{On SaltStack}
\subtitle{Configuration management and remote execution.}

\author{Stas Sușcov (\href{mailto:stas@nerd.ro}{stas@nerd.ro})}
\date{February 23th, 2013}

\institute{GeekMeet \#15, Cluj-Napoca, Transylvania}

\begin{document}

% ----------------------------------------------------------------------------
% *** Titlepage <<<
% ----------------------------------------------------------------------------
\maketitle
% ----------------------------------------------------------------------------
% *** END of Titlepage >>>
% ----------------------------------------------------------------------------

% ----------------------------------------------------------------------------
% *** Frame <<<
% ----------------------------------------------------------------------------
\begin{frame}
\frametitle{About Stas}

\begin{itemize}
  \item a \href{http://stas.nerd.ro}{nerd}
  \item picky developer
  \item interests: web/operations
  \item (\Heart (food wine cycling))
\end{itemize}
\end{frame}
% ----------------------------------------------------------------------------
% *** END of Frame >>>
% ----------------------------------------------------------------------------

% ----------------------------------------------------------------------------
% *** Frame <<<
% ----------------------------------------------------------------------------
\begin{frame}
  \begin{center}
    \huge In August, 2012, I was hired to help migrate one's company monolithic
    infrastructure into the cloud (Linode).
    \\
    \small Lots of experience I am sharing today comes from solving their issues.
  \end{center}
\end{frame}
% ----------------------------------------------------------------------------
% *** END of Frame >>>
% ----------------------------------------------------------------------------

% ----------------------------------------------------------------------------
% *** Frame <<<
% ----------------------------------------------------------------------------
\begin{frame}
\frametitle{The common story}

\begin{itemize}[<+->]
  \item you start with a shared hosting
  \item business grows, you buy more bandwidth and space
  \item business grows, you are thinking to move to dedicated, but you don't, the \emph{works for me} attitude is on
  \item you have a gig with a dozen of employees, you are living a nightmare
  \item development gets slow, knowledge is spread over a couple of people (if you are lucky)
  \item end of story: you are afraid to restart \emph{Apache} because it might not start back!
\end{itemize}
\end{frame}
% ----------------------------------------------------------------------------
% *** END of Frame >>>
% ----------------------------------------------------------------------------

% ----------------------------------------------------------------------------
% *** Frame <<<
% ----------------------------------------------------------------------------
\begin{frame}
\frametitle{Identifying the issue}

\begin{itemize}[<+->]
  \item one server for everything is never \emph{OK}
  \item lack of documentation, writing docs for servers is harder compared to software
  \item lack of a \emph{changelog}, why service \emph{X} runs and service \emph{Y} is stopped
  \item tell me something about this firewall rule (no trolling intended)
  \item lack of deployment tools
  \item lack of provisioning solutions
  \item monitoring should be proactive, graphs are good but still \ldots
  \item lack of an operations-friendly culture (postmortems are for everyone not just your boss)
\end{itemize}
\end{frame}
% ----------------------------------------------------------------------------
% *** END of Frame >>>
% ----------------------------------------------------------------------------

% ----------------------------------------------------------------------------
% *** Frame <<<
% ----------------------------------------------------------------------------
\begin{frame}
  \begin{center}
    \huge We work in an environment where tools reached a level of quality where
    not trusting those, raises lots of questions!
  \end{center}
\end{frame}
% ----------------------------------------------------------------------------
% *** END of Frame >>>
% ----------------------------------------------------------------------------

% ----------------------------------------------------------------------------
% *** Frame <<<
% ----------------------------------------------------------------------------
\begin{frame}
\frametitle{Picking new tools}

Picking new tools is always tricky, you might lose more than win, here are some tips:
\begin{itemize}[<+->]
  \item ask your colleagues developers, you will be impressed to see how many are more than just \emph{programmers}
  \item start picking tools based on current software stack (if you are doing \emph{Python}, look for tools written in that language)
  \item do not judge tools by age, consider facts like documentation, extensibility, development cycle first
  \item last but not least, installation and upgrade actions should be as easy as possible
\end{itemize}
\end{frame}
% ----------------------------------------------------------------------------
% *** END of Frame >>>
% ----------------------------------------------------------------------------

% ----------------------------------------------------------------------------
% *** Frame <<<
% ----------------------------------------------------------------------------
\begin{frame}
\frametitle{What is SaltStack?}

\begin{itemize}[<+->]
  \item SaltStack was designed as a centralized remote execution tool
    \begin{itemize}[<+->]
      \item runs tasks in parallel
      \item uses ØMQ for communication (authenticates using SSH keys)
      \item stand-alone, does not require any other dependencies
    \end{itemize}
  \item SaltStack has an easy to pick configuration management system
    \begin{itemize}[<+->]
      \item configuration management files use an \emph{YAML} syntax
      \item configuration is split into modules and states, which represent pure Python modules
      \item extensible API, overwrite a module by placing the new Python file into local directory (Salt will update machines on its own)
      \item flexible API, ready to use solutions for use-cases like peering, auto-discovery, syndication, white-list execution, returners
    \end{itemize}
\end{itemize}
\end{frame}
% ----------------------------------------------------------------------------
% *** END of Frame >>>
% ----------------------------------------------------------------------------

% ----------------------------------------------------------------------------
% *** Frame <<<
% ----------------------------------------------------------------------------
\begin{frame}
\frametitle{Remote execution}

\begin{itemize}[<+->]
  \item being centralized, service is split between master and minions (slaves)
  \item every salt installation generates an SSH key, that will be used to authenticate the machine
  \item master manages minions/authentication using \texttt{salt-key} tool
  \item master can target minions based on: 
    \begin{itemize}
        \item globbing and regular expressions
        \item static information such as OS, software versions, virtualization, CPU, memory \ldots
        \item statically defined groups
        \item compound matchers
        \item batching execution
    \end{itemize}
\end{itemize}
\end{frame}
% ----------------------------------------------------------------------------
% *** END of Frame >>>
% ----------------------------------------------------------------------------

% ----------------------------------------------------------------------------
% *** Frame <<<
% ----------------------------------------------------------------------------
\begin{frame}[fragile]
  \begin{center}
    \begin{verbatim}
    root@master:~# salt 'slave*' test.ping
                           ^ ^
                     ______| |__________________
                     target  function to execute
    \end{verbatim}
  \end{center}
\end{frame}
% ----------------------------------------------------------------------------
% *** END of Frame >>>
% ----------------------------------------------------------------------------

% ----------------------------------------------------------------------------
% *** Frame <<<
% ----------------------------------------------------------------------------
\begin{frame}
\frametitle{Modules}

\begin{itemize}[<+->]
  \item modules represent functions that \texttt{salt} tool can run on minions
  \item every module is either Python or Cython code
  \item modules can be extended or overwritten by dropping new ones into master \texttt{file\_roots} directory called \texttt{\_modules}
  \item configuration management states, underneath, use modules too, in fact the module name itself is called \texttt{state}
\end{itemize}
\end{frame}
% ----------------------------------------------------------------------------
% *** END of Frame >>>
% ----------------------------------------------------------------------------

% ----------------------------------------------------------------------------
% *** Frame <<<
% ----------------------------------------------------------------------------
\begin{frame}
\frametitle{Configuration management, or state files}

\begin{itemize}[<+->]
  \item SaltStack uses \texttt{YAML} syntax like files called \texttt{SLS} files to describe minion configuration
  \item state files attributes are mapped directly to modules
  \item states can be extended or overwritten by dropping new ones into master \texttt{file\_roots} directory called \texttt{\_states}
  \item states can be grouped using \emph{targeting} in the top file \texttt{top.sls}, and executed using \texttt{state.highstate} call
\end{itemize}
\end{frame}
% ----------------------------------------------------------------------------
% *** END of Frame >>>
% ----------------------------------------------------------------------------

% ----------------------------------------------------------------------------
% *** Frame <<<
% ----------------------------------------------------------------------------
\begin{frame}
  \begin{center}
  \huge Questions please\ldots
  \\
  Thank you for your time.
  \end{center}
\end{frame}
% ----------------------------------------------------------------------------
% *** END of Frame >>>
% ----------------------------------------------------------------------------

% ----------------------------------------------------------------------------
% *** Frame <<<
% ----------------------------------------------------------------------------
\begin{frame}
\frametitle{Online resources worth checking}

\begin{itemize}
  \item SaltStack Documentation -- \href{http://salt.readthedocs.org/en/latest/}{salt.readthedocs.org/en/latest/}
  \item SaltStack Website -- \href{http://saltstack.com/about.html}{saltstack.org}
  \item SaltStack Ops School Chapter -- \href{http://ops-school.readthedocs.org/en/latest/config\_management.html\#saltstack}{ops-school.readthedocs.org/en/latest/config\_management.html\#saltstack}
  \item AppThemes SaltStack -- \href{https://github.com/AppThemes/salt-config-example}{github.com/AppThemes/salt-config-example}
\end{itemize}

\end{frame}
% ----------------------------------------------------------------------------
% *** END of Frame >>>
% ----------------------------------------------------------------------------

\end{document}
