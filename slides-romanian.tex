\documentclass[compress]{beamer}

\beamertemplatetransparentcovered
\setbeamertemplate{navigation symbols}{}
\setbeamertemplate{footline}
{\leavevmode%
  \hbox{%
    \begin{beamercolorbox}[wd=.3333\paperwidth,left]{}%
      \insertshortinstitute
    \end{beamercolorbox}%
    \begin{beamercolorbox}[wd=.3333\paperwidth,center]{}%
      \insertframenumber{} / \inserttotalframenumber
    \end{beamercolorbox}
    \begin{beamercolorbox}[wd=.3333\paperwidth,right]{}%
      \insertshortdate{}
      \;
    \end{beamercolorbox}
  }%
}

\usepackage{listings}
\usepackage{color}
\usepackage{minted}
\usepackage{hyperref}
\usepackage{textcomp}
\usepackage{marvosym}
\usepackage{xcolor}
\usepackage{graphicx}

\usetheme[light,accent=violet]{solarized}

\usepackage{fontspec}
\setmainfont[Mapping=tex-text]{Source Sans Pro}
\setmonofont[Mapping=tex-text]{Inconsolata}

\graphicspath{{./images/}}

\title{Despre SaltStack}
\subtitle{Organizarea configurației și execuție de la distanță.}

\author{Stas Sușcov (\href{mailto:stas@nerd.ro}{stas@nerd.ro})}
\date{23 Februarie, 2013}

\institute{GeekMeet \#15, Cluj-Napoca, Transylvania}

\begin{document}

% ----------------------------------------------------------------------------
% *** Titlepage <<<
% ----------------------------------------------------------------------------
\maketitle
% ----------------------------------------------------------------------------
% *** END of Titlepage >>>
% ----------------------------------------------------------------------------

% ----------------------------------------------------------------------------
% *** Frame <<<
% ----------------------------------------------------------------------------
\begin{frame}
\frametitle{Despre Stas}

\begin{itemize}
  \item a \href{http://stas.nerd.ro}{nerd}
  \item dezvoltator pretențios
  \item interese: web/operations
  \item (\Heart (mâncare vin bicicletă))
\end{itemize}
\end{frame}
% ----------------------------------------------------------------------------
% *** END of Frame >>>
% ----------------------------------------------------------------------------

% ----------------------------------------------------------------------------
% *** Frame <<<
% ----------------------------------------------------------------------------
\begin{frame}
  \begin{center}
    \huge În August, 2012, am fost angajat să reorganizez infrastructura monolitică a unei companii
    în cloud (pe Linode).
    \\
    \small Multe lucruri despre care voi povesti astăzi, ține de experiența obținută în rezolvarea problemelor lor.
  \end{center}
\end{frame}
% ----------------------------------------------------------------------------
% *** END of Frame >>>
% ----------------------------------------------------------------------------

% ----------------------------------------------------------------------------
% *** Frame <<<
% ----------------------------------------------------------------------------
\begin{frame}
\frametitle{O istorie comună}

\begin{itemize}[<+->]
  \item totul începe cu o găzduire partajată
  \item când crește „afacerea”, mai cumperi spațiu și bandă
  \item când mai crește „afacerea”, știi că un serviciu dedicat ar fi mai eficient, însă atitudinea „merge și așa” e greu de schimbat
  \item acum „afacerea” ta are deja 10 oameni și coșmarul abia începe
  \item dezvoltarea încetinește, cunoștințele despre „ce-i pe servere” sunt împrăștiate (actuali și foști angajați)
  \item sfârșit de istorie: îți este frică să repornești \emph{Apache}, că poate nu mai pornește!
\end{itemize}
\end{frame}
% ----------------------------------------------------------------------------
% *** END of Frame >>>
% ----------------------------------------------------------------------------

% ----------------------------------------------------------------------------
% *** Frame <<<
% ----------------------------------------------------------------------------
\begin{frame}
\frametitle{Identificarea problemei}

\begin{itemize}[<+->]
  \item un server pentru orice nu e \emph{OK}
  \item lipsa documentației, e mai greu de scris pentru servere decât pentru software
  \item lipsa unui \emph{istoric}, de ce serviciul \emph{X} rulează, iar serviciul \emph{Y} a fost oprit
  \item regulile de firewall nu spun prea multe
  \item lipsa uneltelor de deployment
  \item lipsa uneltelor de provisioning
  \item uneltele de monitorizare trebuie să fie proactive, graficele nu sunt deajuns \ldots
  \item lipsa unei culturi „operations-friendly” (post-mortem-urile sunt pentru toată lumea, nu doar boși)
\end{itemize}
\end{frame}
% ----------------------------------------------------------------------------
% *** END of Frame >>>
% ----------------------------------------------------------------------------

% ----------------------------------------------------------------------------
% *** Frame <<<
% ----------------------------------------------------------------------------
\begin{frame}
  \begin{center}
    \huge Din fericire, am ajuns să lucrăm în niște medii unde uneltele oferă un
    nivel de calitate înalt, astfel ignoranța folosirii acestora trezește multe întrebări!
  \end{center}
\end{frame}
% ----------------------------------------------------------------------------
% *** END of Frame >>>
% ----------------------------------------------------------------------------

% ----------------------------------------------------------------------------
% *** Frame <<<
% ----------------------------------------------------------------------------
\begin{frame}
\frametitle{Cum alegem uneltele?}

Alegerea uneltelor noi poate fi un lucru dificil, alegerea greșită poate strica mai mult decât ajuta. Ceva sfaturi:
\begin{itemize}[<+->]
  \item povestiți despre unelte cu colegii voștri, veți fi impresionați câți sunt mai mult decât doar \emph{programatori}
  \item alegeți uneltele în funcție de software-ul deja existent/folosit (dacă scrieți \emph{Python}, căutați unelte scrise în acel limbaj)
  \item nu judecați uneltele după vechime, lucrurile cum ar fi documentația, extensibilitatea, ciclul de dezvoltare sunt mult mai importante
  \item nu în ultimul rând, instalarea și actualizarea uneltelor trebuie să fie simplă
\end{itemize}
\end{frame}
% ----------------------------------------------------------------------------
% *** END of Frame >>>
% ----------------------------------------------------------------------------

% ----------------------------------------------------------------------------
% *** Frame <<<
% ----------------------------------------------------------------------------
\begin{frame}
\frametitle{Ce este SaltStack?}

\begin{itemize}[<+->]
  \item SaltStack a fost creat drept o soluție centralizată pentru execuția de la distanță
    \begin{itemize}[<+->]
      \item rulează comenzile în paralel
      \item folosește ØMQ pentru comunicare (se autentifică cu chei SSH)
      \item nu are nevoie de alte dependințe
    \end{itemize}
  \item SaltStack are implementat un sistem simplu de organizarea configurației
    \begin{itemize}[<+->]
      \item fișierele de configurare folosesc o sintaxă \emph{YAML}
      \item principalele componente sunt modulele și stările, care în spate sunt de fapt module pure Python
      \item API extensibil, crearea sau modificarea unei componente se face prin adăugare noului fișier în directorul local (Salt va actualiza singur mașinile)
      \item API flexibil, support pentru diferite organizări cum ar fi peering, auto-discovery, syndication, execuție pe bază de white-list, returners
    \end{itemize}
\end{itemize}
\end{frame}
% ----------------------------------------------------------------------------
% *** END of Frame >>>
% ----------------------------------------------------------------------------

% ----------------------------------------------------------------------------
% *** Frame <<<
% ----------------------------------------------------------------------------
\begin{frame}
\frametitle{Execuția la distanță}

\begin{itemize}[<+->]
  \item fiind centralizat, serviciul este separat între master și minioni („sclavi”)
  \item fiecare instalație de salt generează o cheie SSH, aceasta va fi folosită la autentificare mașinii
  \item pe master se folosește unealta \texttt{salt-key} pentru administrarea „scalvilor” și a cheilor
  \item pe master se pot alege mașinile folosind una din metode: 
    \begin{itemize}
        \item nume sau regex
        \item informații statice cum ar fi SO, versiuni software, virtualizare, CPU, memorie \ldots
        \item grupurilor definite static
        \item compounerea mai multor metode
        \item execuții în serie
    \end{itemize}
\end{itemize}
\end{frame}
% ----------------------------------------------------------------------------
% *** END of Frame >>>
% ----------------------------------------------------------------------------

% ----------------------------------------------------------------------------
% *** Frame <<<
% ----------------------------------------------------------------------------
\begin{frame}[fragile]
  \begin{center}
    \begin{verbatim}
    root@master:~# salt 'slave*' test.ping
                               ^ ^
                    ___________| |________________________
                    ținta aleasă  funcția să fie executată
    \end{verbatim}
  \end{center}
\end{frame}
% ----------------------------------------------------------------------------
% *** END of Frame >>>
% ----------------------------------------------------------------------------

% ----------------------------------------------------------------------------
% *** Frame <<<
% ----------------------------------------------------------------------------
\begin{frame}
\frametitle{Modulele}

\begin{itemize}[<+->]
  \item modulele reprezintă funcții ce pot fi folosite de unealta \texttt{salt} pentru a rula pe minioni
  \item modulele pot fi cod Python sau Cython
  \item pentru module noi, adăugați fișierul pe master în directorul \texttt{file\_roots} cu numele de \texttt{\_modules}
  \item stările folosite la organizarea configurației, folosesc direct modulele, de fapt și stările sunt un modul cu numele \texttt{state}
\end{itemize}
\end{frame}
% ----------------------------------------------------------------------------
% *** END of Frame >>>
% ----------------------------------------------------------------------------

% ----------------------------------------------------------------------------
% *** Frame <<<
% ----------------------------------------------------------------------------
\begin{frame}
\frametitle{Organizarea configurației, sau fișierele de stări}

\begin{itemize}[<+->]
  \item SaltStack folosește fișiere cu sintaxa \texttt{YAML} care se numesc \texttt{SLS}-uri unde se descrie configurația minion-ului
  \item atributele dintr-un fișier de stare se mapează direct la un modul
  \item pentru stări noi, adăugați fișierul pe master în directorul \texttt{file\_roots} cu numele de \texttt{\_states}
  \item stările pot fi grupate pentru a descrie o mașină cu mai multe stări, în fișierul \texttt{top.sls}, după care se apelează funcția \texttt{state.highstate}
\end{itemize}
\end{frame}
% ----------------------------------------------------------------------------
% *** END of Frame >>>
% ----------------------------------------------------------------------------

% ----------------------------------------------------------------------------
% *** Frame <<<
% ----------------------------------------------------------------------------
\begin{frame}
  \begin{center}
  \huge Întrebări\ldots
  \\
  Mulțumesc pentru atenție.
  \end{center}
\end{frame}
% ----------------------------------------------------------------------------
% *** END of Frame >>>
% ----------------------------------------------------------------------------

% ----------------------------------------------------------------------------
% *** Frame <<<
% ----------------------------------------------------------------------------
\begin{frame}
\frametitle{Resurse ce merită să fie verificate}

\begin{itemize}
  \item SaltStack Documentation -- \href{http://salt.readthedocs.org/en/latest/}{salt.readthedocs.org/en/latest/}
  \item SaltStack Website -- \href{http://saltstack.com/about.html}{saltstack.org}
  \item SaltStack Ops School Chapter -- \href{http://ops-school.readthedocs.org/en/latest/config\_management.html\#saltstack}{ops-school.readthedocs.org/en/latest/config\_management.html\#saltstack}
  \item AppThemes SaltStack -- \href{https://github.com/AppThemes/salt-config-example}{github.com/AppThemes/salt-config-example}
\end{itemize}

\end{frame}
% ----------------------------------------------------------------------------
% *** END of Frame >>>
% ----------------------------------------------------------------------------

\end{document}
